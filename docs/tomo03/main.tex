\documentclass[11pt]{article}
\usepackage[a4paper,margin=2.5cm]{geometry}
\usepackage{amsmath,amssymb,siunitx}
\usepackage{hyperref}
\usepackage{cleveref}
\usepackage{csquotes}
\usepackage{graphicx,xcolor}
\usepackage[acronym]{glossaries}
\usepackage[backend=biber,style=authoryear,maxcitenames=2]{biblatex}

% Convenciones y unidades básicas para PDO v1.0
% Sistema natural \hbar = c = k_B = 1
% Signatura métrica: (-,+,+,+)
% Tabla de unidades (por completar por tomo)
\newcommand{\EnergyUnit}{\mathrm{GeV}}
\newcommand{\DensityUnit}{\mathrm{GeV}^4}
\newcommand{\AmpUnit}{\mathrm{GeV}}

\addbibresource{../bib/references.bib}
\makeglossaries
\hypersetup{colorlinks=true, linkcolor=blue, citecolor=blue, urlcolor=blue}
\sisetup{detect-all, per-mode=symbol, separate-uncertainty=true}

\title{Tomo 3 — Ecuación Unificada y Estadística del PDO}
\author{Proyecto PDO}
\date{\today}

\begin{document}
\maketitle
\tableofcontents

\section{Ecuación Unificada}\label{sec:unificada}
Proponemos una representación compacta para las variantes base, \(\eps\) estática y \(\eps\) dinámica:
\begin{equation}
 \mathcal{F}[\psi;\eps] \equiv \mathcal{D}[\psi] + \eps\,\mathcal{O}[\psi] + \mathcal{N}[\psi;\eps] = 0,
 \label{eq:ecuacion-unificada}
\end{equation}
 donde \(\mathcal{D}\) es el operador lineal principal, \(\mathcal{O}\) el término efectivo controlado por \(\eps\) y \(\mathcal{N}\) agrupa no linealidades/perturbaciones.

\section{Observables y Latentaxis}\label{sec:obs}
Usamos la entropía de fase \(S_{\!p}\) como observable operativo (Latentaxis v1), ver \Cref{eq:Sp}.
\begin{equation}
 S_{\!p}(t) = -\sum_i p_i(t)\ln p_i(t), \quad p_i(t)=\frac{|\psi(x_i,t)|^2}{\sum_j |\psi(x_j,t)|^2}.
 \label{eq:Sp}
\end{equation}
El índice \(C(t)=\exp\!\big[-(S_{\!p}(t)-S_{\!p}^{\rm ref})\big]\) cuantifica coherencia.

\section{Estadística de modelos}\label{sec:stat}
Para comparar variantes (con/sin \(\eps\)), reportamos \(\Delta \mathrm{AIC}\) y \(\Delta \mathrm{BIC}\):
\begin{align}
 \mathrm{AIC} &= 2k - 2\ln\hat{L}, &
 \mathrm{BIC} &= k\ln n - 2\ln\hat{L}.
\end{align}
Las decisiones no se basan en “mejor ajuste” aislado sino en evidencia relativa (AIC/BIC).

\section{Bibliografía}
\printbibliography

\end{document}
