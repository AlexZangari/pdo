\documentclass[11pt]{article}
\usepackage[a4paper,margin=2.5cm]{geometry}
\usepackage{amsmath,amssymb,siunitx}
\usepackage{hyperref}
\usepackage{cleveref}
\usepackage{csquotes}
\usepackage{graphicx,xcolor}
\usepackage[acronym]{glossaries}
\usepackage[backend=biber,style=authoryear,maxcitenames=2]{biblatex}

% Convenciones y unidades básicas para PDO v1.0
% Sistema natural \hbar = c = k_B = 1
% Signatura métrica: (-,+,+,+)
% Tabla de unidades (por completar por tomo)
\newcommand{\EnergyUnit}{\mathrm{GeV}}
\newcommand{\DensityUnit}{\mathrm{GeV}^4}
\newcommand{\AmpUnit}{\mathrm{GeV}}

\addbibresource{../bib/references.bib}
\makeglossaries

\sisetup{detect-all, per-mode=symbol, separate-uncertainty=true}
\newacronym{ULDM}{ULDM}{Ultra-Light Dark Matter}

\title{Tomo 2 — Materia Ultraligera y Observables Relacionales}
\author{Proyecto PDO}
\date{\today}

\begin{document}
\maketitle
\tableofcontents

\section{Convenciones y Unidades}
Signatura \(\PDOsign\). Unidades naturales \(\hbar=c=k_B=1\); \(G\) explícito cuando aplique.
Glifo único \(\eps\) para el operador de grieta.

\section{Modelo ULDM}
Sea un campo escalar \(\phi\) con potencial \(V(\phi)\).
La masa \,m_\phi se reportará en \si{\electronvolt} (\textbf{eV}) de forma consistente.
\begin{equation}
\mathcal{L} = \tfrac{1}{2}\partial_\mu\phi\,\partial^\mu\phi - V(\phi).
\label{eq:uldm-lagrangian}
\end{equation}

\section{Escalas y Dimensionalidad}
Usamos \texttt{siunitx}:
\(\,m_\phi = \SI{1e-22}{\electronvolt}\) (ejemplo típico de literatura).
Para conversiones previas en \si{\giga\electronvolt} usar \(1~\si{\giga\electronvolt}=10^9~\si{\electronvolt}\).

\section{Acoplamientos efectivos y \texorpdfstring{\(\eps\)}{epsilon}}
Tratamos \(\eps\) como operador efectivo que modula observables (ver Latentaxis v1 en \Cref{sec:latentaxis}).

\section{Observables}\label{sec:obs}
Matriz de observables: dinámica de fase, espectros, correlaciones.

\section{Latentaxis v1: Entropía de fase \texorpdfstring{\(S_{\!p}\)}{Sp}}
\label{sec:latentaxis}
Sea una discretización espacial \(\{x_i\}\) y una densidad \(p_i(t)\) (normalizada)
asociada al campo \(\psi(x,t)\) o a un funcional derivado. Definimos
\begin{equation}
S_{\!p}(t) \equiv -\sum_i p_i(t)\,\ln p_i(t), \qquad \sum_i p_i(t)=1.
\label{eq:latentaxis-entropy}
\end{equation}
El \textit{índice de coherencia} es \(C(t)=\exp[-\Delta S_{\!p}(t)]\),
con \(\Delta S_{\!p}(t)=S_{\!p}(t)-S_{\!p}^{\,\mathrm{ref}}\).
La hipótesis operativa: \(\eps \uparrow \Rightarrow \Delta S_{\!p}\uparrow\) y \(C\downarrow\).

\printbibliography
\end{document}
