\documentclass[11pt]{article}
\usepackage[a4paper,margin=2.5cm]{geometry}
\usepackage{amsmath,amssymb,siunitx}
\usepackage{hyperref}
\usepackage{cleveref}
\usepackage{csquotes}
\usepackage{graphicx,xcolor}
\usepackage[acronym]{glossaries}
\usepackage[backend=biber,style=authoryear,maxcitenames=2]{biblatex}

% Convenciones y unidades básicas para PDO v1.0
% Sistema natural \hbar = c = k_B = 1
% Signatura métrica: (-,+,+,+)
% Tabla de unidades (por completar por tomo)
\newcommand{\EnergyUnit}{\mathrm{GeV}}
\newcommand{\DensityUnit}{\mathrm{GeV}^4}
\newcommand{\AmpUnit}{\mathrm{GeV}}

\addbibresource{../bib/references.bib}
\makeglossaries
\hypersetup{colorlinks=true, linkcolor=blue, citecolor=blue, urlcolor=blue}
\sisetup{detect-all, per-mode=symbol, separate-uncertainty=true}

\newglossaryentry{eps}{name={\ensuremath{\eps}}, description={Operador de grieta (simetría local efectiva)}}

\title{Tomo 1 — Fundamentos del Paradigma del Origen}
\author{Proyecto PDO}
\date{\today}

\begin{document}
\maketitle
\tableofcontents

\section{Introducción}\label{sec:intro}
Motivación y esquema general del paradigma.

\section{Notación y Unidades}\label{sec:units}
Signatura \(\PDOsign\). Unidades naturales \(\hbar=c=k_B=1\); \(G\) explícito cuando aplique. Glifo único \(\eps\).

\section{Ecuación Base}\label{sec:eqbase}
Definimos la ecuación base del campo \(\psi(x,t)\) y sus términos efectivos.
\begin{equation}
\mathcal{E}[\psi;\eps] = 0.
\label{eq:base}
\end{equation}
Referencia interna a \Cref{eq:base}.

\section{Observables}\label{sec:obs}
Lista de observables y vínculos con \(\eps\). Véase Latentaxis v1 en el Tomo 2.

\section{Glosario}
\printglossaries

\section{Bibliografía}
Citamos trabajos base \cite{Planck2018,Alcubierre1994,EFTbook}.
\printbibliography

\end{document}
