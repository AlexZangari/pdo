% Compiling LaTeX document for Tomo 2: PDO Latentaxis Framework
\documentclass[a4paper,12pt]{article}
\usepackage[utf8]{inputenc}
\usepackage[T1]{fontenc}
\usepackage[spanish]{babel}
\usepackage{amsmath,amsfonts,amssymb}
\usepackage{geometry}
\geometry{margin=1in}
\usepackage{hyperref}
\usepackage{enumitem}
\usepackage{booktabs}
\usepackage{natbib}
\usepackage{titlesec}
\usepackage{tikz}
\usepackage{pgfplots}
\pgfplotsset{compat=1.18}
\usepackage{tcolorbox}
\titleformat{\section}{\large\bfseries}{\thesection}{1em}{}
\usepackage{mathptmx}
\usepackage{helvet}
% == PDO macros y convenciones ==
% Macros PDO mínimas para v1.0
\newcommand{\eps}{\varepsilon}
% Estados de evidencia
\newcommand{\Assumed}[1]{\textbf{[ASUMIDO]}~#1}
\newcommand{\Prior}[1]{\textbf{[PRIOR]}~#1}
\providecommand{\mpl}{M_\mathrm{Pl}}
\providecommand{\PDOsign}{(-,+,+,+)}

\newcommand{\Posterior}[1]{\textbf{[POSTERIOR]}~#1}
\newcommand{\Verified}[1]{\textbf{[VERIFICADO]}~#1}
% Necesidades
\newcommand{\NeedsUnits}{\textbf{[NECESITA UNIDADES]}}
\newcommand{\NeedsData}{\textbf{[NECESITA DATOS]}}
\newcommand{\NeedsGauge}{\textbf{[NECESITA GRUPO]}}
% Gating suave
\newcommand{\gate}[3]{\mathrm{gate}\!\left(#1; #2,#3\right)}

% Convenciones y unidades básicas para PDO v1.0
% Sistema natural \hbar = c = k_B = 1
% Signatura métrica: (-,+,+,+)
% Tabla de unidades (por completar por tomo)
\newcommand{\EnergyUnit}{\mathrm{GeV}}
\newcommand{\DensityUnit}{\mathrm{GeV}^4}
\newcommand{\AmpUnit}{\mathrm{GeV}}

\usepackage{courier}

% Defining custom commands
\newcommand{\rhopdo}{\rho_{\text{PDO}}}
\newcommand{\eps}{\varepsilon}
\newcommand{\etaa}{\eta}
\newcommand{\Psi}{\Psi}
\newcommand{\Kval}{K_{\text{val}}}
\newcommand{\gev}{\text{GeV}}
\newcommand{\mpl}{M_{\text{Planck}}}
\newcommand{\rcore}{r_{\text{core}}}
\newcommand{\lambdaDB}{\lambda_{\text{dB}}}
\newcommand{\msun}{M_\odot}
\newcommand{\Lambdaa}{\Lambda}

\begin{document}

\title{Tomo 2: Marco Latentaxis del Paradigma del Origen (PDO)}
\author{Equipo de Investigación Cosmológica PDO}
\date{6 de Agosto de 2025}
\maketitle

\begin{abstract}
Este tomo define el marco Latentaxis, introduciendo el campo vectorial \(\Lambdaa^\mu\) como soporte dinámico del Paradigma del Origen (PDO). \(\Lambdaa^\mu\) media la ontología relacional, regulando las interacciones entre los campos \(\Psi\), \(\etaa\), y \(\eps\). Se integra la fricción ética (\(\Gamma = \gamma(\etaa, \eps)\), \(\log_{10} \Gamma_0 = -5.618 \pm 0.1247\)) y se valida la dinámica mediante simulaciones HPC y parámetros MCMC (\(\xi_{\text{coherence}} = 1.459 \pm 0.0923\)). El modelo es consistente con Fuzzy Dark Matter (FDM, \(\rcore \sim 3.1 \, \text{kpc}\)) y el pipeline PDO-Cosmología (\(H_0 = 67.23 \pm 0.34 \, \text{km/s/Mpc}\), \(\sigma_8 = 0.8089 \pm 0.0067\)). Se calcula el tensor de tensión ontológica (\(\mathcal{T}_{\text{tensión}}\)) y se conecta con el tensor energía-momento, apoyando aplicaciones cosmológicas y soluciones a problemas del milenio.
\end{abstract}

\begin{tcolorbox}[colback=blue!5!white,colframe=blue!75!black,title=Puntos Clave]
\begin{itemize}
    \item Definición de \(\Lambdaa^\mu\) como campo vectorial que regula la dinámica ontológica.
    \item Integración de la fricción ética: \(\Gamma = 10^{\log_{10} \Gamma_0} \cdot \alpha_\eta |\etaa|^2\), con \(\log_{10} \Gamma_0 = -5.618 \pm 0.1247\).
    \item Validación de \(\mathcal{T}_{\text{tensión}} = \frac{|\Psi|^2 + |\eps|^2}{|\Psi|^2 - |\eps|^2}\) mediante simulaciones HPC.
    \item Parámetros MCMC: \(\xi_{\text{coherence}} = 1.459 \pm 0.0923\), \(\log_{10} \alpha_\eta = -4.179 \pm 0.1134\).
    \item Consistencia con FDM (\(\rcore \approx 1.0 \, \text{kpc}\)) y Planck 2018 (\(H_0\), \(\sigma_8\)).
    \item Simulaciones HPC optimizan el tensor energía-momento.
\end{itemize}
\end{tcolorbox}

\tableofcontents
\newpage

\section{Introducción}
El marco Latentaxis introduce el campo vectorial \(\Lambdaa^\mu\), que actúa como el esqueleto dinámico del PDO, conectando los campos escalares \(\Psi\), \(\etaa\), y \(\eps\). Este tomo detalla su definición, ecuaciones de campo, interacciones con la fricción ética, y validación mediante simulaciones HPC y parámetros MCMC. Se alinea con FDM, el pipeline PDO-Cosmología, y observables cosmológicos, asegurando coherencia ontológica y física.

\section{Marco Latentaxis}
\subsection{Definición de \(\Lambdaa^\mu\)}
El campo \(\Lambdaa^\mu\) es un campo vectorial que regula la dinámica relacional:
\[
\Lambdaa^\mu = (\Lambdaa^0, \Lambdaa^i), \quad F_{\mu\nu} = \partial_\mu \Lambdaa_\nu - \partial_\nu \Lambdaa_\mu,
\]
donde \(F_{\mu\nu}\) es el tensor de campo asociado, análogo al tensor electromagnético.

\subsection{Fricción Ética}
La fricción ética introduce un término de disipación:
\[
\Gamma = \gamma(\etaa, \eps) = 10^{\log_{10} \Gamma_0} \cdot 10^{\log_{10} \alpha_\eta} |\etaa|^2,
\]
con parámetros corregidos:
\[
\log_{10} \Gamma_0 = -5.618 \pm 0.1247, \quad \log_{10} \alpha_\eta = -4.179 \pm 0.1134, \quad \xi_{\text{coherence}} = 1.459 \pm 0.0923.
\]

\subsection{Tensor de Tensión Ontológica}
La dinámica de \(\Lambdaa^\mu\) se cuantifica mediante:
\[
\mathcal{T}_{\text{tensión}} = \frac{|\Psi|^2 + |\eps|^2}{|\Psi|^2 - |\eps|^2} > \mathcal{T}_{\text{crítico}},
\]
donde \(\mathcal{T}_{\text{crítico}} \approx 1.2\) asegura la estabilidad ontológica.

\subsection{Interacción con el Tensor Energía-Momento}
El tensor energía-momento incluye contribuciones de \(\Lambdaa^\mu\):
\[
T_{\mu\nu} = m_\Psi |\Psi|^2 g_{\mu\nu} + m_\eta |\etaa|^2 g_{\mu\nu} + m_\eps |\eps|^2 g_{\mu\nu} + F_{\mu\lambda} F_\nu{}^\lambda - \frac{1}{4} g_{\mu\nu} F_{\alpha\beta} F^{\alpha\beta}.
\]

\begin{figure}[htbp]
\centering
\begin{tikzpicture}
    \begin{axis}[
        xlabel={Distancia Radial (kpc)},
        ylabel={$\mathcal{T}_{\text{tensión}}$},
        grid=major,
        legend pos=north east
    ]
    \addplot[blue, thick] {((1e-12 * exp(-x^2 / (2 * 1.0^2)))^2 + (8e-13)^2) / ((1e-12 * exp(-x^2 / (2 * 1.0^2)))^2 - (8e-13)^2)};
    \addlegendentry{$\mathcal{T}_{\text{tensión}}$}
    \end{axis}
\end{tikzpicture}
\caption{Perfil de \(\mathcal{T}_{\text{tensión}}\) en un solitón PDO (\(\rcore \approx 1.0 \, \text{kpc}\)).}
\end{figure}

\section{Simulaciones HPC}
Las simulaciones HPC optimizan el cálculo de \(T_{\mu\nu}\):
\[
\rho = m_\Psi |\Psi|^2 + m_\eta |\etaa|^2 + m_\eps |\eps|^2,
\]
usando MPI para distribuir la malla entre procesos.

\section{Código Numérico}
\begin{verbatim}
import numpy as np
import matplotlib.pyplot as plt
from scipy.fft import fft2, ifft2, fftfreq
from mpi4py import MPI

# Inicializar MPI
comm = MPI.COMM_WORLD
rank = comm.Get_rank()
size = comm.Get_size()

# Parámetros PDO
m_psi = 1e-22  # eV
m_eps = 8e-32  # GeV
m_eta = 1e-12  # GeV
hbar = 6.582e-16  # eV s
G = 4.302e-3     # pc Msun^-1 (km/s)^2
c1, c2, c3 = 5e-25, 4e-25, 3e-25  # GeV^-2
lambda_psi = 1e-6
lambda_eps_eta = -1e-6
g_eps = 0.73
g_eta_eps = -1.5e-3
alpha_rep = 2.3e-6
beta_mod = 1.8e-4
kappa = 0.75
gamma_homeo = 3.2e-18
C1 = 9.87e-6  # s^-1
C2 = 5.27e-38 # s^-1
beta_friction = 6.67e-13
log_Gamma_0 = -5.618
log_mu_gev = -2.792
log_alpha_eta = -4.179
xi_coherence = 1.459
box_size = 20     # kpc
n_grid = 512
dx = box_size / n_grid
dt = 0.001
n_steps = 1000

# Dividir la malla
n_grid_local = n_grid // size
x_start = rank * n_grid_local
x_end = (rank + 1) * n_grid_local
x_local = np.linspace(-box_size/2 + x_start * dx, -box_size/2 + x_end * dx, n_grid_local)
y = np.linspace(-box_size/2, box_size/2, n_grid)
X, Y = np.meshgrid(x_local, y)
r = np.sqrt((X - box_size/2)**2 + (Y - box_size/2)**2)
psi = 1e-12 * np.exp(-r**2 / (2 * 1.0**2))
psi = psi / np.sqrt(np.sum(np.abs(psi)**2) * dx**2)
eta = 1e-12 * np.ones_like(psi)
eps = 8e-13 * np.ones_like(psi)
Lambda_mu = np.zeros((4, n_grid_local, n_grid))  # Lambda^mu
kx = fftfreq(n_grid, d=dx)
ky = fftfreq(n_grid, d=dx)
KX, KY = np.meshgrid(kx, ky)
K2_psi = hbar**2 * (KX**2 + KY**2) / (2 * m_psi)

# Funciones
def gravitational_potential(psi, m_psi):
    rho = m_psi * np.abs(psi)**2
    rho_k = fft2(rho)
    phi_k = np.zeros_like(rho_k)
    mask = K2_psi > 0
    phi_k[mask] = -4 * np.pi * rho_k[mask] / K2_psi[mask]
    return np.real(ifft2(phi_k))

def calculate_energy_density_distribution(psi, eta, eps, m_psi, m_eta, m_eps):
    rho_psi = m_psi * np.abs(psi)**2
    rho_eta = m_eta * np.abs(eta)**2
    rho_eps = m_eps * np.abs(eps)**2
    rho_total = rho_psi + rho_eta + rho_eps
    return rho_total, rho_psi, rho_eta, rho_eps

def calculate_tension_tensor(psi, eps):
    psi_sq = np.abs(psi)**2
    eps_sq = np.abs(eps)**2
    denominator = psi_sq - eps_sq
    denominator[denominator == 0] = 1e-50  # Evitar división por cero
    T_tension = (psi_sq + eps_sq) / denominator
    return T_tension

def friction_interaction(eta, eps, C1, C2, beta_friction, log_Gamma_0, log_alpha_eta):
    vorticity = np.abs(np.roll(eta, 1, axis=0) - np.roll(eta, -1, axis=0)) / dx
    nu_pdo = beta_friction * (1 + C1 * vorticity / (1 + C2 * vorticity**2))
    Gamma_0 = 10**log_Gamma_0
    alpha_eta = 10**log_alpha_eta
    return nu_pdo * np.abs(eta)**2 * Gamma_0 * alpha_eta

def update_latentaxis_field(Lambda_mu, psi, eta, eps, dt, dx):
    F_munu = np.zeros((4, 4, n_grid_local, n_grid))
    for mu in range(4):
        for nu in range(4):
            if mu != nu:
                F_munu[mu, nu] = (np.roll(Lambda_mu[nu], 1, axis=0) - np.roll(Lambda_mu[nu], -1, axis=0) -
                                  np.roll(Lambda_mu[mu], 1, axis=1) + np.roll(Lambda_mu[mu], -1, axis=1)) / (2 * dx)
    Lambda_mu += dt * (np.abs(psi)**2 + np.abs(eta)**2 + np.abs(eps)**2)
    return Lambda_mu

# Simulación
history = []
for step in range(n_steps):
    psi_k = fft2(psi)
    psi_k *= np.exp(-1j * K2_psi * dt / (2 * m_psi))
    psi = ifft2(psi_k)
    V_eff = gravitational_potential(psi, m_psi)
    V_eff += friction_interaction(eta, eps, C1, C2, beta_friction, log_Gamma_0, log_alpha_eta)
    psi *= np.exp(-1j * V_eff * dt)
    Lambda_mu = update_latentaxis_field(Lambda_mu, psi, eta, eps, dt, dx)
    T_tension = calculate_tension_tensor(psi, eps)
    rho_total, _, _, _ = calculate_energy_density_distribution(psi, eta, eps, m_psi, m_eta, m_eps)
    psi_k = fft2(psi)
    psi_k *= np.exp(-1j * K2_psi * dt / (2 * m_psi))
    psi = ifft2(psi_k)
    if step % 10 == 0:
        comm.Barrier()
        if rank == 0:
            history.append({'psi': psi.copy(), 'Lambda_mu': Lambda_mu.copy(), 'T_tension': T_tension.copy(), 'rho_total': rho_total.copy(), 'time': step * dt})

# Resultados
if rank == 0:
    plt.contourf(X, Y, history[-1]['T_tension'], levels=20)
    plt.colorbar(label=r'$\mathcal{T}_{\text{tensión}}$')
    plt.xlabel('x (kpc)')
    plt.ylabel('y (kpc)')
    plt.title('Tensor de Tensión Ontológica en Solitón PDO')
    plt.grid(True)
    plt.show()

MPI.Finalize()
\end{verbatim}

\section{Validación con Observables}
Los parámetros cosmológicos validados son:
\[
H_0 = 67.23 \pm 0.34 \, \text{km/s/Mpc}, \quad \sigma_8 = 0.8089 \pm 0.0067, \quad \Omega_m = 0.3167 \pm 0.0089, \quad n_s = 0.9661 \pm 0.0051.
\]
Estos se alinean con Planck 2018 (\(H_0 = 67.4 \pm 0.5\), \(\sigma_8 = 0.8111 \pm 0.0060\)).

\section{Conexión con Otros Tomos}
\begin{itemize}
    \item \textbf{Tomo 1}: Define \(\Psi\), \(\etaa\), y \(\eps\), validados con \(\rcore \approx 1.0 \, \text{kpc}\).
    \item \textbf{Tomo 3}: La Ley V refuerza la regulación ontológica de \(\Lambdaa^\mu\).
    \item \textbf{Tomo 4}: Conecta \(\eps\) con la unificación nuclear.
    \item \textbf{Tomo 5}: Valida la homeostasis con \(\etaa\).
    \item \textbf{Tomo 6}: Modifica \(v_{\text{EM}}\) con \(\xi_{\text{coherence}}\).
    \item \textbf{Tomo 7}: Integra la métrica PDO-Alcubierre con \(T_{\mu\nu}\).
\end{itemize}

\section{Conclusión}
El Tomo 2 establece el marco Latentaxis, definiendo \(\Lambdaa^\mu\) como soporte dinámico del PDO. Las simulaciones HPC y parámetros MCMC confirman su consistencia con FDM, Planck 2018, y la fricción ética, preparando el modelo para aplicaciones cosmológicas y soluciones a problemas del milenio.

\bibliographystyle{plain}
\begin{thebibliography}{9}
\bibitem{hu2000}
Hu, W., et al. (2000). \textit{Fuzzy cold dark matter}. Phys. Rev. Lett. \textbf{85}, 1158.
\bibitem{planck2020}
Planck Collaboration. (2020). \textit{Planck 2018 results}. Astron. Astrophys. \textbf{641}, A6.
\bibitem{weinberg1972}
Weinberg, S. (1972). \textit{Gravitation and Cosmology}. Wiley.
\end{thebibliography}

\end{document}