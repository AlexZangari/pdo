% Compiling LaTeX document for Tomo 1: PDO Model Definition
\documentclass[a4paper,12pt]{article}
\usepackage[utf8]{inputenc}
\usepackage[T1]{fontenc}
\usepackage[spanish]{babel}
\usepackage{amsmath,amsfonts,amssymb}
\usepackage{geometry}
\geometry{margin=1in}
\usepackage{hyperref}
\usepackage{enumitem}
\usepackage{booktabs}
\usepackage{natbib}
\usepackage{titlesec}
\usepackage{tikz}
\usepackage{pgfplots}
\pgfplotsset{compat=1.18}
\usepackage{tcolorbox}
\titleformat{\section}{\large\bfseries}{\thesection}{1em}{}
\usepackage{mathptmx}
\usepackage{helvet}
% == PDO macros y convenciones ==
% Macros PDO mínimas para v1.0
\newcommand{\eps}{\varepsilon}
% Estados de evidencia
\newcommand{\Assumed}[1]{\textbf{[ASUMIDO]}~#1}
\newcommand{\Prior}[1]{\textbf{[PRIOR]}~#1}
\newcommand{\Posterior}[1]{\textbf{[POSTERIOR]}~#1}
\newcommand{\Verified}[1]{\textbf{[VERIFICADO]}~#1}
% Necesidades
\newcommand{\NeedsUnits}{\textbf{[NECESITA UNIDADES]}}
\newcommand{\NeedsData}{\textbf{[NECESITA DATOS]}}
\newcommand{\NeedsGauge}{\textbf{[NECESITA GRUPO]}}
% Gating suave
\newcommand{\gate}[2]{\frac{1}{2}\left(1+\tanh\left(\frac{#1-#2}{1}\right)\right)}

% Convenciones y unidades básicas para PDO v1.0
% Sistema natural \hbar = c = k_B = 1
% Signatura métrica: (-,+,+,+)
% Tabla de unidades (por completar por tomo)
\newcommand{\EnergyUnit}{\mathrm{GeV}}
\newcommand{\DensityUnit}{\mathrm{GeV}^4}
\newcommand{\AmpUnit}{\mathrm{GeV}}

\usepackage{courier}

% Defining custom commands
\newcommand{\rhopdo}{\rho_{\text{PDO}}}
\newcommand{\eps}{\varepsilon}
\newcommand{\etaa}{\eta}
\newcommand{\Psi}{\Psi}
\newcommand{\Kval}{K_{\text{val}}}
\newcommand{\gev}{\text{GeV}}
\newcommand{\mpl}{M_{\text{Planck}}}
\newcommand{\rcore}{r_{\text{core}}}
\newcommand{\lambdaDB}{\lambda_{\text{dB}}}
\newcommand{\msun}{M_\odot}
\newcommand{\Lambdaa}{\Lambda}

\begin{document}

\title{Tomo 1: Definición del Modelo del Paradigma del Origen (PDO)}
\author{Equipo de Investigación Cosmológica PDO}
\date{6 de Agosto de 2025}
\maketitle

\begin{abstract}
Este tomo define los fundamentos ontológicos y físicos del Paradigma del Origen (PDO), introduciendo los campos \(\Psi\), \(\etaa\), y \(\eps\), que representan la dinámica relacional, el cuidado, y las grietas ontológicas, respectivamente. Se valida la formación de solitones (\(\rcore \approx 1.0 \, \text{kpc}\)) mediante un solver 2D, integrando la fricción ética (\(\Gamma\)) y parámetros corregidos via MCMC (\(\log_{10} \Gamma_0 = -5.618 \pm 0.1247\)). El modelo es consistente con Fuzzy Dark Matter (FDM, \(\rcore \sim 3.1 \, \text{kpc}\)) y el pipeline PDO-Cosmología (\(H_0 = 67.23 \pm 0.34 \, \text{km/s/Mpc}\)).
\end{abstract}

\begin{tcolorbox}[colback=blue!5!white,colframe=blue!75!black,title=Puntos Clave]
\begin{itemize}
    \item Definición de los campos \(\Psi\), \(\etaa\), y \(\eps\), con masas \(m_\Psi = 10^{-22} \, \text{eV}\), \(m_\eta = 10^{-12} \, \gev\), \(m_\eps = 8 \times 10^{-32} \, \gev\).
    \item Validación de solitones (\(\rcore \approx 1.0 \, \text{kpc}\)) con solver 2D.
    \item Integración de la fricción ética (\(\Gamma = \gamma(\etaa, \eps)\)) para problemas del milenio.
    \item Parámetros corregidos: \(\log_{10} \Gamma_0 = -5.618 \pm 0.1247\), \(\xi_{\text{coherence}} = 1.459 \pm 0.0923\).
    \item Simulaciones HPC optimizan la distribución de densidad de energía (\(\rho\)).
\end{itemize}
\end{tcolorbox}

\tableofcontents
\newpage

\section{Introducción}
El Paradigma del Origen (PDO) propone una ontología relacional que unifica cosmología, física de partículas, y problemas del milenio a través de los campos \(\Psi\), \(\etaa\), y \(\eps\). Este tomo define sus fundamentos, validando solitones y alineándose con FDM y Planck 2018.

\section{Fundamentos del Modelo PDO}
\subsection{Campos Fundamentales}
Los campos del PDO son:
\begin{itemize}
    \item \(\Psi\): Campo escalar ultraligero (\(m_\Psi = 10^{-22} \, \text{eV}\)), modela la dinámica relacional.
    \item \(\etaa\): Campo de cuidado (\(m_\eta = 10^{-12} \, \gev\)), estabiliza sistemas.
    \item \(\eps\): Campo de grietas (\(m_\eps = 8 \times 10^{-32} \, \gev\)), introduce energía/momento.
\end{itemize}

\subsection{Solitones PDO}
El solver 2D genera solitones:
\[
\rcore \approx 1.0 \, \text{kpc}, \quad \lambdaDB = \frac{h}{m_\Psi c} \approx 1.2 \, \text{kpc}.
\]
Comparado con FDM (\(\rcore \sim 3.1 \, \text{kpc}\)).

\subsection{Fricción Ética}
La fricción ética (\(\Gamma = \gamma(\etaa, \eps)\)) introduce un término de disipación:
\[
\Gamma = 10^{\log_{10} \Gamma_0} \cdot \alpha_\eta |\etaa|^2, \quad \log_{10} \Gamma_0 = -5.618 \pm 0.1247, \quad \log_{10} \alpha_\eta = -4.179 \pm 0.1134.
\]

\begin{figure}[htbp]
\centering
\begin{tikzpicture}
    \begin{axis}[
        xlabel={Distancia Radial (kpc)},
        ylabel={Densidad de Energía ($M_\odot$ kpc$^{-3}$)},
        grid=major,
        ymode=log,
        legend pos=north east
    ]
    \addplot[blue, thick] {1e-12 * exp(-x^2 / (2 * 1.0^2))};
    \addlegendentry{$\rho_\Psi$}
    \end{axis}
\end{tikzpicture}
\caption{Perfil de densidad del solitón PDO (\(\rcore \approx 1.0 \, \text{kpc}\)).}
\end{figure}

\section{Código Numérico}
\begin{verbatim}
import numpy as np
import matplotlib.pyplot as plt
from scipy.fft import fft2, ifft2, fftfreq
from mpi4py import MPI

# Inicializar MPI
comm = MPI.COMM_WORLD
rank = comm.Get_rank()
size = comm.Get_size()

# Parámetros PDO
m_psi = 1e-22  # eV
m_eps = 8e-32  # GeV
m_eta = 1e-12  # GeV
hbar = 6.582e-16  # eV s
G = 4.302e-3     # pc Msun^-1 (km/s)^2
c1, c2, c3 = 5e-25, 4e-25, 3e-25  # GeV^-2
lambda_psi = 1e-6
lambda_eps_eta = -1e-6
g_eps = 0.73
g_eta_eps = -1.5e-3
alpha_rep = 2.3e-6
beta_mod = 1.8e-4
kappa = 0.75
gamma_homeo = 3.2e-18
C1 = 9.87e-6  # s^-1
C2 = 5.27e-38 # s^-1
beta_friction = 6.67e-13
log_Gamma_0 = -5.618
log_mu_gev = -2.792
log_alpha_eta = -4.179
xi_coherence = 1.459
box_size = 20     # kpc
n_grid = 512
dx = box_size / n_grid
dt = 0.001
n_steps = 1000

# Dividir la malla
n_grid_local = n_grid // size
x_start = rank * n_grid_local
x_end = (rank + 1) * n_grid_local
x_local = np.linspace(-box_size/2 + x_start * dx, -box_size/2 + x_end * dx, n_grid_local)
y = np.linspace(-box_size/2, box_size/2, n_grid)
X, Y = np.meshgrid(x_local, y)
r = np.sqrt((X - box_size/2)**2 + (Y - box_size/2)**2)
psi = 1e-12 * np.exp(-r**2 / (2 * 1.0**2))
psi = psi / np.sqrt(np.sum(np.abs(psi)**2) * dx**2)
eta = 1e-12 * np.ones_like(psi)
eps = 8e-13 * np.ones_like(psi)
kx = fftfreq(n_grid, d=dx)
ky = fftfreq(n_grid, d=dx)
KX, KY = np.meshgrid(kx, ky)
K2_psi = hbar**2 * (KX**2 + KY**2) / (2 * m_psi)

# Funciones
def gravitational_potential(psi, m_psi):
    rho = m_psi * np.abs(psi)**2
    rho_k = fft2(rho)
    phi_k = np.zeros_like(rho_k)
    mask = K2_psi > 0
    phi_k[mask] = -4 * np.pi * rho_k[mask] / K2_psi[mask]
    return np.real(ifft2(phi_k))

def calculate_energy_density_distribution(psi, eta, eps, m_psi, m_eta, m_eps):
    rho_psi = m_psi * np.abs(psi)**2
    rho_eta = m_eta * np.abs(eta)**2
    rho_eps = m_eps * np.abs(eps)**2
    rho_total = rho_psi + rho_eta + rho_eps
    return rho_total, rho_psi, rho_eta, rho_eps

# Simulación
history = []
for step in range(n_steps):
    psi_k = fft2(psi)
    psi_k *= np.exp(-1j * K2_psi * dt / (2 * m_psi))
    psi = ifft2(psi_k)
    V_eff = gravitational_potential(psi, m_psi)
    rho_total, _, _, _ = calculate_energy_density_distribution(psi, eta, eps, m_psi, m_eta, m_eps)
    psi *= np.exp(-1j * V_eff * dt)
    psi_k = fft2(psi)
    psi_k *= np.exp(-1j * K2_psi * dt / (2 * m_psi))
    psi = ifft2(psi_k)
    if step % 10 == 0:
        comm.Barrier()
        if rank == 0:
            history.append({'psi': psi.copy(), 'rho_total': rho_total.copy(), 'time': step * dt})

# Resultados
if rank == 0:
    rho = m_psi * np.abs(psi)**2
    plt.contourf(X, Y, rho, levels=20)
    plt.colorbar(label=r'$\rho$ ($M_\odot$ kpc$^{-3}$)')
    plt.xlabel('x (kpc)')
    plt.ylabel('y (kpc)')
    plt.title('Perfil de Densidad del Solitón PDO')
    plt.grid(True)
    plt.show()

MPI.Finalize()
\end{verbatim}

\section{Conexión con Otros Tomos}
\begin{itemize}
    \item \textbf{Tomo 2}: Extiende la dinámica ontológica con \(\Lambdaa^\mu\).
    \item \textbf{Tomo 3}: Refuerza la complementariedad ser-grieta.
    \item \textbf{Tomo 4}: Conecta \(\eps\) con la unificación nuclear.
    \item \textbf{Tomo 5}: Valida la homeostasis con simulaciones HPC.
    \item \textbf{Tomo 6}: Modifica \(v_{\text{EM}}\) con \(\xi_{\text{coherence}}\).
    \item \textbf{Tomo 7}: Integra la métrica PDO-Alcubierre y WARP-DRIVE.
\end{itemize}

\section{Conclusión}
El Tomo 1 establece los fundamentos del PDO, validando solitones (\(\rcore \approx 1.0 \, \text{kpc}\)) y la fricción ética, alineándose con FDM y Planck 2018.

\bibliographystyle{plain}
\begin{thebibliography}{9}
\bibitem{hu2000}
Hu, W., et al. (2000). \textit{Fuzzy cold dark matter}. Phys. Rev. Lett. \textbf{85}, 1158.
\bibitem{planck2020}
Planck Collaboration. (2020). \textit{Planck 2018 results}. Astron. Astrophys. \textbf{641}, A6.
\end{thebibliography}

\end{document}